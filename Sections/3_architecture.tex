
\chapter{L'architecture calculateur} \label{CHAP3}
\smallskip
\hfill
\begin{minipage}[b]{8cm}
{\it Il est tr\`es important, pour celui qui souhaite d\'ecouvrir, de ne pas limiter son esprit \'a un seul chapitre de la
science mais plut\^ot de rester en contact avec plusieurs autres.}
\end{minipage}
\begin{flushright} Jacques Hadamard. \end{flushright}
\vskip 2cm


\section{Section Title}
{\Huge L}orem ipsum dolor sit amet, consectetur adipisicing elit, sed do eiusmod tempor incididunt ut labore et dolore magna aliqua. Ut enim ad minim veniam, quis nostrud exercitation ullamco laboris nisi ut aliquip ex ea commodo consequat. Duis aute irure dolor in reprehenderit in voluptate velit esse cillum dolore eu fugiat nulla pariatur. Excepteur sint occaecat cupidatat non proident, sunt in culpa qui officia deserunt mollit anim id est laborum.

\begin{figure}
    \centering
    \begin{subfigure}[b]{0.3\textwidth}
        \centering
        \includegraphics[width=\textwidth]{graph1}
        \caption{$y=x$}
        \label{fig:y equals x}
    \end{subfigure}
    \hfill
    \begin{subfigure}[b]{0.3\textwidth}
        \centering
        \includegraphics[width=\textwidth]{graph2}
        \caption{$y=3sinx$}
        \label{fig:three sin x}
    \end{subfigure}
    \hfill
    \begin{subfigure}[b]{0.3\textwidth}
        \centering
        \includegraphics[width=\textwidth]{graph3}
        \caption{$y=5/x$}
        \label{fig:five over x}
    \end{subfigure}
    \caption{Three simple graphs}
    \label{fig:three graphs}
\end{figure}

