
\chapter{La th\'eorie des jeux} \label{CHAP3}
\smallskip
\hfill
\begin{minipage}[b]{8cm}
{\it Il est tr\`es important, pour celui qui souhaite d\'ecouvrir, de ne pas limiter son esprit \'a un seul chapitre de la
science mais plut\^ot de rester en contact avec plusieurs autres.}
\end{minipage}
\begin{flushright} Jacques Hadamard. \end{flushright}
\vskip 2cm


\section{Pr\'esentation}

La th\'eorie des jeux est la science de la prise de d\'ecision strat\'egique. La théorie des jeux a été utilisée dans des sciences aussi diverses que la biologie évolutionniste, le management des d\'ecisions (politique) et l'économie. La th\'eorie des jeux peut \^etre d\'efinie comme l'\'etude de mod\`eles math\'ematiques de conflit et coop\'eration entre d\'ecideurs intelligents et rationnels.
La th\'eorie des jeux fournit des techniques math\'ematiques g\'en\'erales pour analyser des situations dans lesquelles deux personnes ou plus prennent des d\'ecisions qui s'influencent mutuellement.

Dans le langage de la th\'eorie des jeux, un jeu fait r\'ef\'erence \`a toute situation sociale impliquant deux sujets ou plus. Les sujets impliqu\'es dans un jeu peuvent \^etre appel\'es les \emph{joueurs}. Comme indiqu\'e dans la d\'efinition ci-dessus, il y a deux hypoth\`eses de base que les th\'eoriciens des jeux font g\'en\'eralement \`a propos des joueurs: ils sont rationnels et intelligents. 
Chacun de ces adjectifs est utilis\'e ici dans un sens technique qui n\'ecessite quelques explications. Un d\'ecideur est rationnel s'il prend des d\'ecisions de mani\`ere coh\'erente de ses propres objectifs. En th\'eorie des jeux, en s'appuyant sur le fondamental r\'esultats de la théorie de la d\'ecision, nous supposons que l'objectif de chaque joueur est de maximiser la valeur attendue de son propre gain, qui est mesur\'e en une certaine \'echelle d'utilit\'e. L'id\'ee qu'un décideur rationnel devrait faire les d\'ecisions qui maximiseront ses b\'en\'efices d'utilit\'e attendus remontent au moins \`a Bernoulli (1738), mais la justification moderne de cette id\'ee tient \`a Von Neumann et Morgenstern (1947).


Bla bla bla