
\chapter{La s\'ecurit\'e dans le monde de l'automobile} \label{CHAP2}
\smallskip
\hfill
\begin{minipage}[b]{8cm}
{\it Ce travail f\^ut pr\'esent\'e en partie \`a la conf\'erence des sourd-muets unijambistes \`a Quib\'eron en avril 1994.}
\end{minipage}
\begin{flushright} Maxime Ayrault \end{flushright}
\vskip 2cm

\section {La r\'evolution logicielle}
\medskip
{\Huge L}e logiciel embarqu\'e est une des innovations cl\'e dans le monde automobile.
D'apr\`es l'article de Robert Charette (\cite{Char09}) paru dans IEEE Spectrum, la premi\`ere voiture embarquant un logiciel \'etait la Oldsmobile Toronado de General Motors en 1977. La Toronado avait une unit\'e de contr\^ole \'electronique (ECU) qui g\'erait la synchronisation des bougies (spark timing). En 1978, General Mortors proposait en option sur ses Cadillacs un ordinateur de bord qui pouvait afficher la vitesse, le niveau du r\'eservoir d'essence, les informations sur l'\'etat du v\'ehicule. Ce logicel s'ex\'ecutait sur une version modif\'ee du processeur Motorola 6802 et faisait 50.000 lignes de code. Depuis, de plus en plus de fonctions ont \'et\'e r\'elis\'ees par du logiciel. Afin de limiter le nobre de cables dans une voiture, les capteurs ont \'et\'e d\'ecentrali\'ses et connect\'es \`a un r\'eseau interne (r\'eseau CAN). Le logiciel a \'egalement permis de cr\'eer de nouvelles fonctions pour les voitures. \\
Une voiture actuelle est maintenant devenue une plateforme logiciel sur roues. Une voiture grand public contient entre 30 to 50 unit\'es de contr\^ole \'electronique effectuant la gestion de multiples syst\`emes (voir \ref{tab:soft}).

\FloatBarrier
\begin{table}
\centering
\begin{tabular}{| l | l | l |}
\hline
Air bag & ABS & Syst\`eme d'alarme \\
\hline
La climatisation & Le r\'egulateur de vitesse & Le r\'egime moteur \\
\hline
Les clignotants & Les feux & Le klaxon \\
\hline
La gestion des si\`eges & Le syst\`eme de navigation & Le syste\`eme audio \\
\hline
la pression des pneus & les vitres & ... \\
\hline
\end{tabular}
\caption{Syst\`eme logiciel dans une voiture}
\label{tab:soft}
\end{table}
\FloatBarrier

Le co\^ut du logiciel et de l'\'electronique repr\'esente entre 35 \`a 40\% du co\^ut d'une voiture.



\section {Le risque cybers\'ecurit\'e}
 \medskip
 {\Huge L}'ajout de logiciels et de connectivit\'e....
