
\chapter{La s\'ecurit\'e dans le monde de l'automobile} \label{CHAP2}
\smallskip
\hfill
\begin{minipage}[b]{8cm}
{\it Ce travail f\^ut pr\'esent\'e en partie \`a la conf\'erence des sourds-muets unijambistes \`a Quib\'eron en avril 1994.}
\end{minipage}
\begin{flushright} Maxime Ayrault \end{flushright}
\vskip 2cm

\section {La r\'evolution logicielle}
\medskip
{\Huge L}e logiciel embarqu\'e est une des innovations cl\'e dans le monde automobile.
D'apr\`es l'article de Robert Charette (\cite{Cha09}) paru dans IEEE Spectrum, la premi\`ere voiture embarquant un logiciel \'etait la Oldsmobile Toronado de General Motors en 1977. La Toronado avait une unit\'e de contr\^ole \'electronique (ECU) qui g\'erait la synchronisation des bougies (spark timing). En 1978, General Motors proposait en option sur ses Cadillacs un ordinateur de bord qui pouvait afficher la vitesse, le niveau du r\'eservoir d'essence, les informations sur l'\'etat du v\'ehicule. Ce logiciel s'ex\'ecutait sur une version modifi\'ee du processeur Motorola 6802 et faisait 50.000 lignes de code. Depuis, de plus en plus de fonctions ont \'et\'e r\'ealis\'ees par du logiciel. Afin de limiter le nombre de c\^ables dans une voiture, les capteurs ont \'et\'e d\'ecentralis\'es et connect\'es \`a un r\'eseau interne (r\'eseau CAN). Le logiciel a \'egalement permis de cr\'eer de nouvelles fonctions pour les voitures. \\
Une voiture actuelle est maintenant devenue une plate-forme logiciel sur roues. Une voiture grand public contient entre 30 et 50 unit\'es de contr\^ole \'electronique effectuant la gestion de multiples syst\`emes (voir \ref{tab:soft}).

\FloatBarrier
\begin{table}
\centering
\begin{tabular}{| l | l | l |}
\hline
Air bag & ABS & Syst\`eme d'alarme \\
\hline
La climatisation & Le r\'egulateur de vitesse & Le r\'egime moteur \\
\hline
Les clignotants & Les feux & Le klaxon \\
\hline
La gestion des si\`eges & Le syst\`eme de navigation & Le syst\`eme audio \\
\hline
la pression des pneus & les vitres & ... \\
\hline
\end{tabular}
\caption{Syst\`eme logiciel dans une voiture}
\label{tab:soft}
\end{table}
\FloatBarrier


En 2009, Alfred Katzenbach, le directeur des Technologies Informatique Chez Daimler, a annon\c c\'e
que le syst\`eme navigation et audio sur une Mercedes-Benz  S-class contient plus 20
millions de lignes de code et que la voiture contient pratiquement autant d'unit\'e \'electronique qu'un Airbus A380 (si on exclut le syst\`eme de divertissement en vol) (\cite{Cha09}).  Les logiciels dans une voiture grossissent \`a une vitesse exponentielle en taille et en complexit\'e. En 2010, certaines voitures avaient 10 millions de lignes de code; an dix ans, ceci a augment\'e par un facteur 15, pour environ 150 millions de lignes. Aujourd'hui, le Model S de Tesla est \'equip\'ee d'un \'ecran tactile de 17 pouces bas\'e sur un syst\`eme d'exploitation Linux qui contr\^ole quasiment toute les fonctions de la voiture, de la performance moteur au syst\`eme audio et de navigation. En fait, il n'y a plus que 2 boutons manuel; le bouton pour les feux de d\'etresse et le bouton de la bo\^ite \`a gants. \\
Aujourd'hui, le co\^ut du logiciel et de l'\'electronique repr\'esente entre 35 \`a 40\% du co\^ut d'une voiture. \\
Il n'est pas rare que les manuels utilisateurs des voitures fassent maintenant plus de 500 pages pour expliquer l'ensemble des fonctions logiciel. On estime qu'un conducteur moyen n'utilise pas plus de 20% des fonctions impl\'ement\'ees par du logiciel. 
Ceci a aussi d'importantes cons\'equences sur la fa\c con d'entretenir et de r\'eparer une voiture. On estime que plus de 50% des unit\'es \'electroniques remplac\'ees n'ont pas d'erreur (logiciel ou mat\'eriel). Le garagiste remplace la pi\`ece simplement car il ne sait pas quelle est la cause principales de la panne. La principale activit\'e des garagiste consiste \`a t\'e\'echarger les nouvelles versions du logiciel. Sur les voitures Tesla, les mises \`a jour logiciel, qui incluent les corrections du logiciel et les nouvelles fonctionnalit\'es sont t\'el\'echarg\'ees dans la voiture via le r\'eseau cellulaire sans intervention d'un garagiste.\\  



\begin{tbd}
Introduire Automotive Open System Architecture (AUTOSAR). Une solution pour r\'eduire le co\^ut du logiciel en augmentant l'interop\'erabilit\'e entre les constructeurs.
\end{tbd}


L'introduction du logiciel a permis d'avoir des voitures plus s\^urs et moins polluantes mais a \'egalement introduit une nouvelle menace; la \emph{s\'ecurit\'e des voitures}. 


\section {Le risque cybers\'ecurit\'e}
 \medskip
 {\Huge L}'ajout de logiciels et de connectivit\'e....


\begin{tbd}
\url{https://www.nytimes.com/2015/09/27/business/complex-car-software-becomes-the-weak-spot-under-the-hood.html}
Fear of Hacking
Andy Greenberg steered a 2014 white Jeep Cherokee down a highway in St. Louis, cruising along at 70 miles per hour. Miles away, two local hackers, Charlie Miller and Chris Valasek, sat on a leather couch at Mr. Miller's house, laptops open, ready to wreak havoc.

As Mr. Greenberg sped along, both hands on the wheel, his ride began to go awry. First, the air-conditioning began blasting. Then an image of the hackers in tracksuits appeared on the digital display screen. Rap music began blaring at full volume, and Mr. Greenberg could not adjust the sound. The windshield wipers started and cleaning fluid sprayed, obstructing his view. Finally, the engine quit.
Mr. Greenberg was on a highway with no shoulder. A big rig blew past, blaring its horn.

"I'm going to pull over," Mr. Greenberg said. " Because I have PTSD."

The episode was in fact a stunt orchestrated by the hackers and Mr. Greenberg, a writer for Wired magazine, to demonstrate the Jeep's very real vulnerabilities. The article appeared on July 21.

Days later, Fiat Chrysler, the maker of Jeep, announced a recall of 1.4 million vehicles to fix the flaws the hackers had identified - the first known recall intended to address a possible hacking threat.
\end{tbd}
